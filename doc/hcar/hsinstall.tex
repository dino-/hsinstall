% hsinstall.tex
\begin{hcarentry}[new]{hsinstall}
\report{Dino Morelli}
\status{stable, actively developed}
\makeheader

\vspace{5mm}

This is a utility to install Haskell programs on a system using
stack. Although stack does have an `install` command, it only copies
binaries. Sometimes more is needed, other files and some directory
structure. hsinstall tries to install the binaries, the LICENSE
file and also the resources directory if it finds one.

\vspace{5mm}

Installations can be performed in one of two directory
structures. FHS, or the Filesystem Hierarchy Standard (most UNIX-like
systems) and what I call "bundle" which is a portable directory
for the app and all of its files. They look like this:

\vspace{5mm}

bundle is sort-of a self-contained structure like this:

\vspace{5mm}

\begin{verbatim}
     $PREFIX/
       $PROJECT-$VERSION/
         bin/...
         doc/LICENSE
         resources/...
\end{verbatim}

\vspace{5mm}

fhs is the more traditional UNIX structure like this:

\vspace{5mm}

\begin{verbatim}
     $PREFIX/
       bin/...
       share/
         $PROJECT-$VERSION/
           doc/LICENSE
           resources/...
\end{verbatim}

\vspace{5mm}

There are two parts to hsinstall that are intended to work 
together. The first part is a Haskell shell script,
`util/install.hs`. Take a copy of this script and check it into
a project you're working on. This will be your installation
script. Running the script with the `--help` switch will explain
the options. Near the top of the script are default values for
these options that should be tuned to what your project needs.

\vspace{5mm}

The other part of hsinstall is a library. The install script will try
to install a `resources` directory if it finds one. the HSInstall
library can then be used in your code to locate the resources
at runtime.

\vspace{5mm}

Note that you only need the library if your software has data files
it needs to locate at runtime in the installation directories. Many
programs don't have this requirement and can ignore the library
altogether.

\vspace{5mm}

Source code is available on darcshub, Hackage and Stackage

\FurtherReading
\begin{compactitem}
 \item hsinstall on darcshub \url{http://hub.darcs.net/dino/hsinstall}
 \item hsinstall on Hackage \url{https://hackage.haskell.org/package/hsinstall}
 \item hsinstall on Stackage \url{https://www.stackage.org/package/hsinstall}
\end{compactitem}

\end{hcarentry}
